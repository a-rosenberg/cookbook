\PassOptionsToPackage{unicode=true}{hyperref} % options for packages loaded elsewhere
\PassOptionsToPackage{hyphens}{url}
%
\documentclass[]{book}
\usepackage{lmodern}
\usepackage{amssymb,amsmath}
\usepackage{ifxetex,ifluatex}
\usepackage{fixltx2e} % provides \textsubscript
\ifnum 0\ifxetex 1\fi\ifluatex 1\fi=0 % if pdftex
  \usepackage[T1]{fontenc}
  \usepackage[utf8]{inputenc}
  \usepackage{textcomp} % provides euro and other symbols
\else % if luatex or xelatex
  \usepackage{unicode-math}
  \defaultfontfeatures{Ligatures=TeX,Scale=MatchLowercase}
\fi
% use upquote if available, for straight quotes in verbatim environments
\IfFileExists{upquote.sty}{\usepackage{upquote}}{}
% use microtype if available
\IfFileExists{microtype.sty}{%
\usepackage[]{microtype}
\UseMicrotypeSet[protrusion]{basicmath} % disable protrusion for tt fonts
}{}
\IfFileExists{parskip.sty}{%
\usepackage{parskip}
}{% else
\setlength{\parindent}{0pt}
\setlength{\parskip}{6pt plus 2pt minus 1pt}
}
\usepackage{hyperref}
\hypersetup{
            pdftitle={Cookbook Book},
            pdfborder={0 0 0},
            breaklinks=true}
\urlstyle{same}  % don't use monospace font for urls
\usepackage{longtable,booktabs}
% Fix footnotes in tables (requires footnote package)
\IfFileExists{footnote.sty}{\usepackage{footnote}\makesavenoteenv{longtable}}{}
\usepackage{graphicx,grffile}
\makeatletter
\def\maxwidth{\ifdim\Gin@nat@width>\linewidth\linewidth\else\Gin@nat@width\fi}
\def\maxheight{\ifdim\Gin@nat@height>\textheight\textheight\else\Gin@nat@height\fi}
\makeatother
% Scale images if necessary, so that they will not overflow the page
% margins by default, and it is still possible to overwrite the defaults
% using explicit options in \includegraphics[width, height, ...]{}
\setkeys{Gin}{width=\maxwidth,height=\maxheight,keepaspectratio}
\setlength{\emergencystretch}{3em}  % prevent overfull lines
\providecommand{\tightlist}{%
  \setlength{\itemsep}{0pt}\setlength{\parskip}{0pt}}
\setcounter{secnumdepth}{5}
% Redefines (sub)paragraphs to behave more like sections
\ifx\paragraph\undefined\else
\let\oldparagraph\paragraph
\renewcommand{\paragraph}[1]{\oldparagraph{#1}\mbox{}}
\fi
\ifx\subparagraph\undefined\else
\let\oldsubparagraph\subparagraph
\renewcommand{\subparagraph}[1]{\oldsubparagraph{#1}\mbox{}}
\fi

% set default figure placement to htbp
\makeatletter
\def\fps@figure{htbp}
\makeatother

\usepackage{booktabs}
\usepackage[]{natbib}
\bibliographystyle{apalike}

\title{Cookbook Book}
\author{}
\date{\vspace{-2.5em}2020-01-04}

\begin{document}
\maketitle

{
\setcounter{tocdepth}{1}
\tableofcontents
}
\hypertarget{aloo-gobi}{%
\chapter{aloo gobi}\label{aloo-gobi}}

2.6.2018

aloo gobi

ingredients \textbar{}
instructions \textbar{}
notes

\hypertarget{ingredients}{%
\subsection{ingredients}\label{ingredients}}

\begin{itemize}
\tightlist
\item
  1 head cauliflower; cut into big size florets
\item
  1-2 russet potato, about 1 lb; large dice
\item
  1-2 Tbsp neutral oil
\item
  1" ginger root; peeled and minced
\item
  1 serrano; small dice, deseeded if preferred
\item
  3-4 on the vine tomatoes; diced
\item
  1 cup green peas
\item
  1 tsp ground cumin
\item
  1 tsp garam masala
\item
  2 tsp coriander
\item
  1/2 tsp turmeric
\item
  1/4 tsp cayenne
\item
  1/2 bunch cilantro; rinsed, rough chop
\item
  1 lemon (optional)
\item
  1 recipe; perfect basmati
\end{itemize}

\hypertarget{instructions}{%
\subsection{instructions}\label{instructions}}

\begin{enumerate}
\def\labelenumi{\arabic{enumi}.}
\tightlist
\item
  Preheat the oven to 400°F. Toss cauliflower florets and potatoes with a couple tbsp neutral oil. Spread on baking sheet.
  Roast until caramelized and mostly tender, 20-30 minutes, tossing halfway through.
\item
  Heat a couple tbsp neutral oil in small stainless-steel skillet until shimmering. Add ginger and serrano chilis and
  cook, stirring frequently, until fragrant. About 1 minute. Add cumin, garam masala, coriander, turmeric, cayenne, and
  pinch of salt. Allow spices to toast in oil, being careful not to burn, for about one minute. Add
  roasted cauliflower and potato and toss well in spices. Cook for an additional 2 minutes, then add tomatoes. Cook until
  thickened and fragrant, about 5 minutes more.
\item
  Add green peas and cook until warmed through - test some to make sure they're good.
  Adjust seasoning to taste with salt, lemon and additional garam masala or cayenne.
\item
  Stir in cilantro and serve with rice, passing lemon wedges tableside.
\end{enumerate}

\hypertarget{notes}{%
\subsection{notes}\label{notes}}

Would work well with roti or naan too. Check out steamed basmati recipe.

\hypertarget{lemon-red-pepper-quinoa-salad}{%
\chapter{lemon red pepper quinoa salad}\label{lemon-red-pepper-quinoa-salad}}

9.6.2018

lemon red pepper quinoa salad

ingredients \textbar{}
instructions \textbar{}
notes

\hypertarget{ingredients-1}{%
\subsection{ingredients}\label{ingredients-1}}

\begin{itemize}
\tightlist
\item
  1 cup dry quinoa (about 3 cups cooked)
\item
  1 3/4 cup water
\item
  1/2 tsp kosher salt
\item
  1 tsp olive oil or butter
\item
  1 cucumber; small dice
\item
  1 red pepper; small dice
\item
  1 small red onion; small dice (maybe 1/2 or 3/4 of a giant one)
\item
  1 can chickpeas; rinsed and drained
\item
  1 bunch of parsley, finely chopped
\item
  1/4 cup olive oil
\item
  1/4 lemon juice (2 or 3 lemons)
\item
  1 Tbsp sherry vinegar (red wine vinegar could work too)
\item
  2 garlic cloves; pressed
\item
  1/2 tsp kosher salt
\end{itemize}

\hypertarget{instructions-1}{%
\subsection{instructions}\label{instructions-1}}

\begin{enumerate}
\def\labelenumi{\arabic{enumi}.}
\tightlist
\item
  Cook the quinoa. Thoroughly rinse dry quinoa while heating oil in medium sized pot over medium heat. When quinoa
  are washed and pan hot, add drained quinoa and cook stirring to toast as the remaining water evaporates - about 2-3
  minutes. Stir in water and salt and bring to a rolling boil. Reduce heat to the lowest setting and cover the pot.\\
  Cook for 15 minutes without removing the top, then take off heat and let sit, unopened, for 5 more minutes.\\
  Fluff with a fork and allow to cool some.
\item
  While quinoa cooks, prep and combine the ingredients from cucumber to parsley in a large mixing bowl.
\item
  Make the dressing. Combine ingredients from olive oil to kosher salt in a mason jar. Put on the lid and give it a good shake.
\item
  Once the quinoa has cooled a bit as to not cook the veggies, add to the mixing bowl and stir to mix. Shake the dressing again and pour
  over the mixture while stirring. It's good now a little warm, and will be good cold out of the fridge for another few days.
\end{enumerate}

\hypertarget{notes-1}{%
\subsection{notes}\label{notes-1}}

Adapted from a cookie and kate blog recipe. I like the sherry vinegar a lot more than red wine in this application. As I was making
it I thought it would need something like feta, but once I tasted it I really don't think that's the case. Dressing might benefit from
a pinch of red pepper flakes. A good camping recipe due to the short cook and cold presentation.

\hypertarget{lentil-bolognese}{%
\chapter{lentil bolognese}\label{lentil-bolognese}}

11.4.2017

lentil bolognese

ingredients \textbar{}
instructions \textbar{}
notes

\hypertarget{ingredients-2}{%
\subsection{ingredients}\label{ingredients-2}}

\begin{itemize}
\tightlist
\item
  3 Tbsp olive oil (not extra virgin)
\item
  2 carrots; small dice
\item
  1-2 celery stalks; small dice
\item
  1 onion; small dice
\item
  1 cup beluga (black) lentils
\item
  4 cups vegetable broth
\item
  2 Tbsp unsalted butter
\item
  1 scant Tbsp balsamic vinegar
\item
  pasta, dried or homemade; cooked
\end{itemize}

\hypertarget{instructions-2}{%
\subsection{instructions}\label{instructions-2}}

\begin{enumerate}
\def\labelenumi{\arabic{enumi}.}
\tightlist
\item
  Heat the oil over medium-low heat then add carrots. Cook for a couple minutes then add onions and celery and cook
  until golden; about 10-15 minutes.
\item
  Rinse lentils then add to the pot. Toss around then add the 4 cups of broth. Bring to a boil over high heat then
  reduce to a simmer and
  allow to cook for 35-45 minutes until the lentils have absorbed much of the water and are abundantly tender. This is a
  good time to check the seasoning.
\item
  Remove half the sauce to a blender or food processor and blitz to a rough paste. Add this back to the pot along with
  butter and the balsamic vinegar.
\item
  Serve over noodles with a few cracks of fresh pepper and some chopped parsley.
\end{enumerate}

\hypertarget{notes-2}{%
\subsection{notes}\label{notes-2}}

Can use any type of broth. Mushroom or chicken broth would be nice. For boxed pasta I prefer \emph{De Cecco} brand and a
shape without holes or tubes for the sauce to get stuck in. Good options are farfalle, oriechette or fettucine. Also
goes really well over roasted spaghetti squash.

\hypertarget{mediterranean_couscous_salad}{%
\chapter{mediterranean\_couscous\_salad}\label{mediterranean_couscous_salad}}

5.20.2017

mediterranean couscous salad

ingredients \textbar{}
instructions \textbar{}
notes

\hypertarget{ingredients-3}{%
\subsection{ingredients}\label{ingredients-3}}

\begin{itemize}
\tightlist
\item
  1.5 cup whole wheat couscous
\item
  1 english cucumber; small dice
\item
  1 small carton good cherry tomatoes; cut in half
\item
  1 red pepper; small dice (optional, as a replacement for tomatoes)
\item
  1 can garbanzo beans; drained
\item
  1/4 cup (or more) italian dressing
\item
  1 avocado (optional as topping)
\item
  salt
\end{itemize}

\hypertarget{instructions-3}{%
\subsection{instructions}\label{instructions-3}}

\begin{enumerate}
\def\labelenumi{\arabic{enumi}.}
\tightlist
\item
  Boil 1.5 cups of water with a pinch of salt in a small saucepan. Once it's boiling, add couscous and stir, take
  off heat and let rest for 5+ minutes with the lid on before fluffing with a fork.
\item
  Combine veggies and dressing to let marinate while couscous steams.
\item
  Top couscous with veggie salad. Top that with optional sliced avocados. Add more dressing if necessary.
\end{enumerate}

\hypertarget{notes-3}{%
\subsection{notes}\label{notes-3}}

If really getting fancy, could add chopped parsley. If using persian cucumbers prep 2-3 instead of 1. For chickpeas,
I prefer \emph{Goya} brand.

\hypertarget{miso-braised-greens}{%
\chapter{miso braised greens}\label{miso-braised-greens}}

10.3.2017

miso braised greens

ingredients \textbar{}
instructions \textbar{}
notes

\hypertarget{ingredients-4}{%
\subsection{ingredients}\label{ingredients-4}}

\begin{itemize}
\tightlist
\item
  2 Tbsp neutral oil
\item
  1 bunch swiss chard; sliced, stems removed and chopped.
\item
  3 cloves garlic; smashed under flat knife blade then thinly sliced
\item
  2 Tbsp miso paste, light or medium
\item
  1 tsp rice vinegar
\item
  1 pinch red pepper flakes (optional)
\end{itemize}

\hypertarget{instructions-4}{%
\subsection{instructions}\label{instructions-4}}

\begin{enumerate}
\def\labelenumi{\arabic{enumi}.}
\tightlist
\item
  Cook stems in oil over medium low heat until the stems are starting to cook but still firm.
\item
  Add garlic (and red pepper flakes if using) and simmer while stirring for another minute. Add chard leaves and
  stir while cooking until begins to release water.
\item
  Add miso paste and rice vinegar and stir to evenly combine into the chard water. If it's too dry add a couple
  tablespoons. Cook until most water evaporates and the chard is in a sticky sauce.
\item
  Season to taste and serve. Add more miso or vinegar as needed.
\end{enumerate}

\hypertarget{notes-4}{%
\subsection{notes}\label{notes-4}}

Chard is easily replaced with lacinato (aka dinosaur) kale. Cut out and discard stems. The rest of the directions
remain the same.

\hypertarget{miso-rice-bowl}{%
\chapter{miso rice bowl}\label{miso-rice-bowl}}

12.31.2017

miso rice bowl

ingredients \textbar{}
instructions \textbar{}
notes \textbar{}

\hypertarget{ingredients-5}{%
\subsection{ingredients}\label{ingredients-5}}

\begin{itemize}
\tightlist
\item
  3/4 cup of brown rice
\item
  1/4 cup of wild rice
\item
  1-2 sweet potatoes; peeled and 3/4" cubed
\item
  1 large bundle of broccoli
\item
  2-4 tbsp neutral oil
\item
  coarse or kosher salt
\item
  freshly ground black pepper
\item
  1 tsp white sesame seeds
\item
  1 tsp black sesame seeds
\item
  1 recipe miso-tahini dressing
\end{itemize}

\hypertarget{instructions-5}{%
\subsection{instructions}\label{instructions-5}}

\begin{enumerate}
\def\labelenumi{\arabic{enumi}.}
\tightlist
\item
  Heat oven to 400F. Cook rice in a rice cooker on the brown setting.
\item
  Toss sweet potatoes in a neutral oil and a sprinkle of salt.
\item
  Using one large or two smaller trays, spread out sweet potatoes and roast for 20 minutes, until browning underneath.
  Flip, toss, and add broccoli to the tray. Drizzle with a little more neutral oil and a sprinkle of salt. Roast for
  another 10-20 minutes. Broccoli should be lightly charred at the edges and sweet potatoes should be browned and tender.
\item
  While vegetables are roasting, prepare the miso-tahini dressing.
\item
  Assemble bowls. Scoop rice into each bowl, pile on vegetables, and coat lightly with dressing. Sprinkle toasted
  sesame seeds on top.
\end{enumerate}

\hypertarget{notes-5}{%
\subsection{notes}\label{notes-5}}

Easily doubled. Equally delicious over wheat berries or farro. Also good with a drizzle of sriracha.

\hypertarget{miso-tahini-dressing}{%
\chapter{miso tahini dressing}\label{miso-tahini-dressing}}

12.31.2017

miso tahini dressing

ingredients \textbar{}
instructions \textbar{}
notes

\hypertarget{ingredients-6}{%
\subsection{ingredients}\label{ingredients-6}}

\begin{itemize}
\tightlist
\item
  1 Tbsp fresh ginger; chopped
\item
  1 small garlic clove; chopped
\item
  2 Tbsp light (shiro) yellow miso
\item
  2 Tbsp tahini
\item
  1 scant Tbsp honey (or maple syrup)
\item
  1/4 cup rice vinegar
\item
  2 Tbsp toasted sesame oil
\item
  2 Tbsp olive oil
\end{itemize}

\hypertarget{instructions-6}{%
\subsection{instructions}\label{instructions-6}}

\begin{enumerate}
\def\labelenumi{\arabic{enumi}.}
\tightlist
\item
  Combine everything in a blender and run until smooth, scraping down sides once.
\item
  Taste and adjust as needed.
\end{enumerate}

\hypertarget{notes-6}{%
\subsection{notes}\label{notes-6}}

Stores well. This is an acceptably sized small batch, but there isn't much reason not to double this recipe.

\hypertarget{morning-smoothie}{%
\chapter{morning smoothie}\label{morning-smoothie}}

5.29.2017

morning banana smoothie

ingredients \textbar{}
instructions \textbar{}
notes

\hypertarget{ingredients-7}{%
\subsection{ingredients}\label{ingredients-7}}

\begin{itemize}
\tightlist
\item
  1 peeled; previously frozen banana
\item
  1 cup unsweetened plain almond milk
\item
  2 Tbsp natural peanut butter
\item
  1 Tbsp rolled oats
\item
  1/2 cup frozen spinach or 1 to 2 cups of fresh, rinsed spinach (optional)
\item
  1 tsp flax seed (optional)
\item
  1 tsp spirulina powder (optional)
\item
  1 pinch garam masala or cinnamon powder (optional)
\end{itemize}

\hypertarget{instructions-7}{%
\subsection{instructions}\label{instructions-7}}

\begin{enumerate}
\def\labelenumi{\arabic{enumi}.}
\tightlist
\item
  Add to blender in the general order the ingredients are listed. If you want to really mix it up add one of the `optional' spices.
\item
  Blend until smooth, using the tamper or adding almond milk to help everything combine.
\end{enumerate}

\hypertarget{notes-7}{%
\subsection{notes}\label{notes-7}}

Can be made with more or less almond milk to adjust the consistency between a shake/ice cream treat or a drink. I
sometimes add parsley if it's around.

\hypertarget{mung-bean-korean-pancake-batter}{%
\chapter{mung bean korean pancake batter}\label{mung-bean-korean-pancake-batter}}

11.3.2019

mung bean korean pancake batter

ingredients \textbar{}
instructions \textbar{}
notes

\hypertarget{ingredients-8}{%
\subsection{ingredients}\label{ingredients-8}}

\begin{itemize}
\tightlist
\item
  1 cup split yellow mung beans
\item
  2.5 cups of water; separated
\end{itemize}

\hypertarget{instructions-8}{%
\subsection{instructions}\label{instructions-8}}

\begin{enumerate}
\def\labelenumi{\arabic{enumi}.}
\tightlist
\item
  Place the beans and 2 cups of the water in a medium bowl and
  soak for 1 hour. They should increase in volume by about 50 percent.
\item
  Drain the beans, then place in a blender. Add the remaining 1/2 cup
  water. Blend on high speed until the mixture is very smooth, 15 to 30 seconds.
\item
  Season before cooking. Do not add salt until ready to cook.
\end{enumerate}

\hypertarget{notes-8}{%
\subsection{notes}\label{notes-8}}

You can make this mixture and refrigerate, covered, for up to 3 days. Can add kimchi, sauerkraut, green onions, sprouted
mung beans, tamari, fish sauce, etc.

\hypertarget{mushroom-risotto}{%
\chapter{mushroom risotto}\label{mushroom-risotto}}

5.1.2017

pressure cooker mushroom risotto

ingredients \textbar{}
instructions \textbar{}
notes

\hypertarget{ingredients-9}{%
\subsection{ingredients}\label{ingredients-9}}

\begin{itemize}
\tightlist
\item
  4 cups vegetable stock
\item
  1.5 lbs assorted mushrooms (a mixture is best); cleaned, trimmed and thinly sliced
\item
  4 Tbsp olive oil
\item
  4 Tbsp butter
\item
  1 yellow onion; chopped
\item
  4 garlic cloves; minced
\item
  1.5 cup arborio rice
\item
  2 tsp tamari or soy sauce
\item
  1 Tbsp light (shiro) miso
\item
  1/2 cup white wine
\item
  1 oz grated high quality parmesan
\item
  more parmesan and any finely minced herb for serving
\item
  1 oz dried porcini or shitake (optional)
\end{itemize}

\hypertarget{instructions-9}{%
\subsection{instructions}\label{instructions-9}}

\begin{enumerate}
\def\labelenumi{\arabic{enumi}.}
\tightlist
\item
  If you have dried mushrooms, heat stock and allow dried mushrooms to steep until soft, about 5 minutes for
  porcinis or up to an hour for shitakes. Remove hydrated mushrooms and roughly chop. Add mushroom scraps to the stock
  and allow to sit until later when the stock is added to the rice.
\item
  Heat olive and oil and butter in the pressure cooker and add all of the mushrooms. Cook until the moisture is
  released and the mushrooms are beginning to brown.
\item
  Add onion and garlic and cook until barely softened, then add rice and cook until starting to become translucent
  around the edges; about 5 minutes.
\item
  Stir in soy/tamari and miso paste until distributed then add wine and cook until the alcohol is released; about
  2-3 minutes.
\item
  Pour the stock into the pot through a mesh strainer and discard the mushroom scraps. Cook on low pressure for 5
  minutes (10 psi on most cookers). Release steam valve and open the pressure cooker.
\item
  Stir in parmesan and check the seasoning.
\item
  Serve with extra grated parmesan and herbs at the table.
\end{enumerate}

\hypertarget{notes-9}{%
\subsection{notes}\label{notes-9}}

Not a good choice for someone who doesn't like mushrooms. The dried mushrooms are nice but not necessary - I
generally skip the extra step. Also, if the rissotto comes out a little too dry, feel free to loosen it with a
few tablespoons of stock or heavy cream.

\hypertarget{mustard-arugula-farm-salad}{%
\chapter{mustard arugula farm salad}\label{mustard-arugula-farm-salad}}

12.27.2018

mustard arugula farm salad

ingredients \textbar{}
instructions \textbar{}
notes

\hypertarget{ingredients-10}{%
\subsection{ingredients}\label{ingredients-10}}

\begin{itemize}
\tightlist
\item
  2 Tbsp dijon mustard
\item
  2 Tbsp whole grain mustard (e.g., Maille)
\item
  1/2 cup extra virgin olive oil
\item
  2 Tbsp sherry vinegar (or something similar)
\item
  1 scant Tbsp maple syrup
\item
  couple cracks of black pepper
\item
  12-16 oz new potatoes; cooked (boiled or steamed) and cut in half
\item
  7 oz arugula; washed and dried as much as possible
\item
  1 1/4 cup uncooked brown rice
\item
  1/4 cup uncooked wild rice
\item
  1/4 cup dried cranberries
\item
  4-6 eggs; quartered lengthwise
\item
  1/4 cup pine nuts (optional)
\item
  1 tin sardines or mackerel (optional)
\end{itemize}

\hypertarget{instructions-10}{%
\subsection{instructions}\label{instructions-10}}

\begin{enumerate}
\def\labelenumi{\arabic{enumi}.}
\tightlist
\item
  Combine all the ingredients in the top section in a pint jar with a lid (mason jar). Shake to create an even mixture then put in the fridge.
\item
  Cook the brown and wild rice together. I like to combine them in a rice cooker and cook via timer overnight to be ready in the morning.
\item
  Combine potatoes, arugula, rice, pine nuts and cranberries and toss until evenly distributed. Serve onto plates or into containers
  and top with eggs and fish -- should make 4-6 meal sized servings. When ready to serve, drizzle over two spoonfuls of dressing. Add more to taste.
\end{enumerate}

\hypertarget{notes-10}{%
\subsection{notes}\label{notes-10}}

Fish isn't necessary but a nice boost of protein.

\hypertarget{cilantro-lime-burritos}{%
\chapter{cilantro lime burritos}\label{cilantro-lime-burritos}}

12.20.2018

cilantro lime burritos

ingredients \textbar{}
instructions \textbar{}
notes

\hypertarget{ingredients-11}{%
\subsection{ingredients}\label{ingredients-11}}

\begin{itemize}
\tightlist
\item
  2.5 cups dry short grain brown rice
\item
  2 bay leaves
\item
  1 lime
\item
  1/2 cup cilantro; chopped
\item
  1 pinch salt
\item
  1-2 Tbsp neutral oil
\item
  tortillas, large, about 10"
\item
  2 sweet potatoes; small to medium dice
\item
  2 cans black beans with liquid
\item
  2 garlic cloves; lightly crushed
\item
  3 springs cilantro; whole
\item
  shredded cheese (pepper-jack is nice)
\item
  red salsa
\item
  hot sauce (optional)
\item
  nutritional yeast (optional)
\end{itemize}

\hypertarget{instructions-11}{%
\subsection{instructions}\label{instructions-11}}

\begin{enumerate}
\def\labelenumi{\arabic{enumi}.}
\tightlist
\item
  Cook short grain rice in rice cooker with bay leaves and correct amount of water. Once finished, remove leaves, add juice of 1/2 lime, cilantro, salt and neutral oil. Taste the rice and adjust as necessary.
\item
  Combine whole cans of beans, crushed garlic and cilantro springs in a pot over medium low heat. Simmer for 30 minutes until super tender then crush some (maybe 1/4) of the beans against the side of the pot to release starch. Stir well.
\item
  Preheat oven to 400F. Toss sweet potatoes in neutral oil and a pinch of salt and lay on a parchment lined baking sheet. Roast for 12 minutes then flip and roast another 12, checking a couple times during cooking to prevent any burning.
\item
  Heat tortilla until it is supple and almost too hot to handle. Lay on a plate and pile on the rice, then beans, then sweet potato. Top with salsa and cheese. Burrito roll it and serve hot.
\end{enumerate}

\hypertarget{notes-11}{%
\subsection{notes}\label{notes-11}}

If not serving immediately, keep the main ingredients in the fridge and assemble each burrito when ready for cooking.

\hypertarget{open-faced-breakfast-sammy}{%
\chapter{open faced breakfast sammy}\label{open-faced-breakfast-sammy}}

3.8.2017

open faced breakfast sammy

ingredients \textbar{}
instructions \textbar{}
notes

\hypertarget{ingredients-12}{%
\subsection{ingredients}\label{ingredients-12}}

\begin{itemize}
\tightlist
\item
  1 slice of sourdough bread
\item
  1 cup of uncooked spinach
\item
  2 thin slices of ``unexpected cheddar'' (Trader Joe's)
\item
  1 egg
\item
  1-2 Tbsp neutral oil
\item
  1 tsp olive oil
\item
  kosher salt
\item
  ground pepper
\end{itemize}

\hypertarget{instructions-12}{%
\subsection{instructions}\label{instructions-12}}

\begin{enumerate}
\def\labelenumi{\arabic{enumi}.}
\tightlist
\item
  Toast sourdough bread.
\item
  While bread is toasting, heat pan to medium heat and add 1-2 Tbsp of neutral oil to the pan. Once heated, add
  spinach to one side of pan and crack an egg on the other side.
\item
  Allow egg to cook until the white congeals and the inside is somewhere between runny and molten. While egg is
  cooking, make sure to toss spinach to ensure it cooks evenly.
\item
  Assemble sandwich. Drizzle olive oil over bread. Layer the cheese, spinach, and fried egg. Add a sprinkle of salt
  and a dash of pepper.
\end{enumerate}

\hypertarget{notes-12}{%
\subsection{notes}\label{notes-12}}

Delicious served on own or with a side of fruit.

\hypertarget{pasta-e-ceci}{%
\chapter{pasta e ceci}\label{pasta-e-ceci}}

7.11.2019

pasta e ceci (pasta and chickpea soup)

ingredients \textbar{}
instructions \textbar{}
notes

\hypertarget{ingredients-13}{%
\subsection{ingredients}\label{ingredients-13}}

\begin{itemize}
\tightlist
\item
  3 Tbsp olive oil
\item
  2 stalks celery; roughly chopped
\item
  1 carrot; roughly chopped
\item
  1 yellow onion; roughly chopped
\item
  3 sprigs rosemary; minced
\item
  6 cups vegetable stock (I prefer better than bouillon)
\item
  1 15 oz cans chickpeas; drained and rinsed
\item
  8 oz. small pasta (cavatelli, ditalini or oriechette)
\item
  kosher salt and freshly ground black pepper, to taste
\item
  2 Tbsp parsley; minced
\item
  parmesan cheese for serving
\item
  lemon (optional)
\end{itemize}

\hypertarget{instructions-13}{%
\subsection{instructions}\label{instructions-13}}

\begin{enumerate}
\def\labelenumi{\arabic{enumi}.}
\tightlist
\item
  Heat oil in a 6-qt. saucepan over medium-high; add rosemary, celery, carrot, and onion and cook until soft, 8-10 minutes.
\item
  Add stock and chickpeas; simmer 5 minutes. Remove half the chickpeas and purée until smooth; return chickpeas to pan.
\item
  Add pasta and cook until al dente, 10 minutes; season with salt and pepper. Taste for acid and add lemon juice if needed.
\item
  Stir in parsley and serve with parmesan cheese.
\end{enumerate}

\hypertarget{notes-13}{%
\subsection{notes}\label{notes-13}}

Doubles well. If you don't have fresh rosemary, do not add dried. Alternatively, consider adding fresh thyme.

\hypertarget{rustic-baguette}{%
\chapter{rustic baguette}\label{rustic-baguette}}

9.27.2018

rustic baguettes

ingredients \textbar{}
instructions \textbar{}
notes

\hypertarget{ingredients-14}{%
\subsection{ingredients}\label{ingredients-14}}

\begin{itemize}
\tightlist
\item
  680 grams lukewarm water; less than 100F
\item
  1 Tbsp granulated yeast
\item
  1 1/2 Tbsp kosher salt
\item
  680 g white flour
\item
  230 g whole wheat flour
\item
  some semolina flour
\end{itemize}

\hypertarget{instructions-14}{%
\subsection{instructions}\label{instructions-14}}

\begin{enumerate}
\def\labelenumi{\arabic{enumi}.}
\tightlist
\item
  In a large, sealable container over a scale add water Then yeast and salt. Mix a little bit.\\
  Tare then add white flour, tare again and add whole wheat. Combine thoroughly with a danish dough
  whisk or wooden spoon, making sure to get the corners.
\item
  Let sit out, with lid on lightly so that gases can escape, for 2-4 hours while it rises then collapses a bit.\\
  Then place in the fridge for 3 hours to 10 days.
\item
  When ready to cook, preheat oven to 450F with a steamer tray. Dust bin top with a little semolina then gently
  shape into baguettes, rolling in a little more semolina as needed to prevent sticking. Let rest for 20 minutes then
  slash 3 times lengthwise on a slight angle and bake for 25 minutes.
\item
  Let cool, then eat. Should last a day or two.
\end{enumerate}

\hypertarget{notes-14}{%
\subsection{notes}\label{notes-14}}

Makes 5-8 baguettes depending on size.

\hypertarget{salt-brine-pickles}{%
\chapter{salt brine pickles}\label{salt-brine-pickles}}

1.2.2020

salt brine fermented pickles

ingredients \textbar{}
instructions \textbar{}
notes

\hypertarget{ingredients-15}{%
\subsection{ingredients}\label{ingredients-15}}

\begin{itemize}
\tightlist
\item
  2 TB salt
\item
  vegetables
\end{itemize}

\hypertarget{instructions-15}{%
\subsection{instructions}\label{instructions-15}}

\begin{enumerate}
\def\labelenumi{\arabic{enumi}.}
\tightlist
\item
  Stir 2 TB salt into a 1 quart mason jar until salt is dissolved.
\item
  Add chopped vegetables to the mason jar. Turnips, beets, radishes, carrots are particularly tasty.
\item
  Optionally, add spices, particularly garlic, ginger, peppers (hot or sweet), and onions.
\item
  Cover the mason jar loosely with its lid. If possible, ensure that the vegetables are submerged with the brine: one easy way to do this is to fill a plastic bag with beans and place it on top of the vegetables and brine. Specialized tools also exist. If the vegetables are not completely covered, those that are exposed to the air can be removed before storage.
\item
  Ferment between 2-3 days and 2-3 weeks, or longer. Refrigerate to store with a closed lid.
\end{enumerate}

\hypertarget{notes-15}{%
\subsection{notes}\label{notes-15}}

If possible, use filtered water, as chlorine in water may inhibit lactic acid fermentation. The ratio of 2 TB salt to 1 quart of water can be scaled up or down (i.e., for 1 gallon of water, use 8 TB {[}1/2 cup{]} of salt). Salt can be reduced to taste, especially when a longer time is used for the ferment. Especially in cooler weather, as little as 1 TB salt per quart of water may suffice. This technique works for any vegetable, though leafy greens are known to taste bitter with this method.

\hypertarget{shakshuka}{%
\chapter{shakshuka}\label{shakshuka}}

12.31.2017

shakshuka

ingredients \textbar{}
instructions \textbar{}
notes \textbar{}

\hypertarget{ingredients-16}{%
\subsection{ingredients}\label{ingredients-16}}

\begin{itemize}
\tightlist
\item
  1 medium onion; small dice
\item
  4 cloves garlic; crushed then sliced
\item
  2 Anaheim peppers; seeds removed and diced
\item
  3 jalapeño or 2 serrano peppers; sliced
\item
  1 Tbsp cumin
\item
  2 tsp smoked paprika
\item
  28oz san marzano tomatoes
\item
  6 or 7 eggs
\item
  1 handful of chopped parsley
\item
  1/4 cup crumbled feta
\item
  olive oil and pepper to finsih
\end{itemize}

\hypertarget{instructions-16}{%
\subsection{instructions}\label{instructions-16}}

\begin{enumerate}
\def\labelenumi{\arabic{enumi}.}
\tightlist
\item
  Heat a 12-14" pan on medium heat. Add onions, peppers, a few tablespoons of olive oil and a small pinch of salt and
  cook until onions are translucent and just beginning to brown.
\item
  Add the garlic and cook for 30 seconds. Add the cumin and smoked paprika and cook for 30 more. Dump in the tomatoes
  and crush with a wooden spoon until it resembles a chunky pasta sauce. Bring to a boil then reduce to low and simmer
  for approximately 15 minutes, stirring occasionally to prevent the sauce from burning at the bottom, until it is the
  consistency of oatmeal.
\item
  This is a good time to taste the sauce and season it until it's right.
\item
  Create egg divots with the back of a spoon and distribute the eggs into the divots. Spoon the sauce over some of the
  egg white before covering. Cook for 5-10 minutes checking for the eggs to be completely cooked like it is poached.
\item
  Once the eggs are just about cooked, remove from heat and sprinkle with feta and parsley, a light drizzle of a good
  olive and a few cracks of pepper.
\end{enumerate}

\hypertarget{notes-16}{%
\subsection{notes}\label{notes-16}}

Goes well with baguette or naan. Can add lots of different veggies to the base: carrots, radishes, green beans,
eggplant, etc. Also goes well with a drizzle of tahini sauce.

\hypertarget{steamed-basmati-rice}{%
\chapter{steamed basmati rice}\label{steamed-basmati-rice}}

2.9.2018

steamed basmati rice

ingredients \textbar{}
instructions \textbar{}
notes

\hypertarget{ingredients-17}{%
\subsection{ingredients}\label{ingredients-17}}

\begin{itemize}
\tightlist
\item
  1 cup basmati rice
\item
  1.5 cups just boiled water
\item
  1-2 Tbsp neutral oil
\item
  1 pinch salt (3/4 tsp salt)
\item
  spices: cloves, cinnamon stick, black peppercorns, cumin seeds, mustard seeds, curry leaves (all optional)
\end{itemize}

\hypertarget{instructions-17}{%
\subsection{instructions}\label{instructions-17}}

\begin{enumerate}
\def\labelenumi{\arabic{enumi}.}
\tightlist
\item
  Rinse rice under cold water a few times or until it runs clear. Cover in cold water and let sit for at
  least 20 minutes, ideally 40 minutes, up to one hour.
\item
  Heat oil in a broad pan (at least 12-14"). If using spices, add to the hot oil and fry until fragrant.
\item
  Add drained rice and the salt and toss to coat. Dump in just boiling water and bring pan to an aggresive boil.
\item
  Once boiling, cover, turn heat down to low and let cook for 10 minutes.
\item
  Keeping top on, remove from heat and let steam in residual heat for another 10 minutes.
\item
  Serve with tasty curries.
\end{enumerate}

\hypertarget{notes-17}{%
\subsection{notes}\label{notes-17}}

Some combinations of aromatics I enjoy are a cinnamon stick and cloves, or cumin or mustard seeds and curry leaves.
Can be scaled 100\% when doubling, but reduce to 2.5 cups boiling water.

\hypertarget{sweet-potato-bisque}{%
\chapter{sweet potato bisque}\label{sweet-potato-bisque}}

12.21.2019

sweet potato bisque

ingredients \textbar{}
instructions \textbar{}
notes

\hypertarget{ingredients-18}{%
\subsection{ingredients}\label{ingredients-18}}

\begin{itemize}
\tightlist
\item
  2 Tbsp coconut oil (or other neutral oil)
\item
  1 onion or 2 shallots; diced
\item
  1 thumb ginger; finely chopped
\item
  2 lb sweet potato; peeled and cubed
\item
  1.5 tsp salt
\item
  1 Tbsp curry powder
\item
  1 tsp ground coriander
\item
  3 cups water
\item
  1/2 cup coconut milk (or almond milk) plus more for drizzling if desired
\item
  1/4 bunch of cilantro (optional); washed and chopped
\item
  1 lime (optional); cut into wedges
\end{itemize}

\hypertarget{instructions-18}{%
\subsection{instructions}\label{instructions-18}}

\begin{enumerate}
\def\labelenumi{\arabic{enumi}.}
\tightlist
\item
  Cook the onion and ginger in coconut oil in a deep soup pot over medium heat stirring occasionally, until soft -- about
  10 minutes.
\item
  Add the salt and spices and stir. Add the cubed sweet potatoes and toss to coat then add the water. Bring to a boil
  then reduce to a simmer and cook until fork tender -- about 15 minutes.
\item
  Transfer to a blender and remove blender plug then cover with a towel (or paper towel that will resist yellowing
  from curry powder). Blend until smooth. Either manually increasing slowly from low to high speed remaining on high for
  about 30 seconds or using the puree setting on the Vitamix.
\item
  Rinse out pot if any small pieces of onion or potato remain and transfer puree back. Reheat over a low flame
  and stir in coconut milk (or almond milk). Check the seasoning then serve with cilantro and lime.
\end{enumerate}

\hypertarget{notes-18}{%
\subsection{notes}\label{notes-18}}

Also goes well with some crunchy toasted bread.

\hypertarget{turmeric-immune-soup}{%
\chapter{turmeric immune soup}\label{turmeric-immune-soup}}

11.5.2018

turmeric immune soup

ingredients \textbar{}
instructions \textbar{}
notes

\hypertarget{ingredients-19}{%
\subsection{ingredients}\label{ingredients-19}}

\begin{itemize}
\tightlist
\item
  2 tsp olive oil
\item
  4 carrots; small dice
\item
  4 celery stalks; small dice
\item
  2 smallish white, sweet or spanish onion; small dice
\item
  2-3 inches garlic; minced
\item
  6-8 garlic cloves; minced
\item
  3 tsp ground turmeric
\item
  1/2 tsp red chili flakes
\item
  1 1/3 cup dried orzo
\item
  8 cups vegetable broth
\item
  4 tsp light miso
\item
  2 tsp apple cider vinegar
\item
  1/4 cup parsley, flat or curly; chopped
\end{itemize}

\hypertarget{instructions-19}{%
\subsection{instructions}\label{instructions-19}}

\begin{enumerate}
\def\labelenumi{\arabic{enumi}.}
\tightlist
\item
  Heat a broad based pot over medium high heat. Add olive oil, then onion, carrot and celery. Cook until onion is
  very translucent, about 5-10 minutes. Add garlic, ginger, turmeric and chili flakes and cook until fragrant, about a
  minute. Add broth and bring to boil.
\item
  Once at a low boil, add orzo and cook stirring intermittently until almost al dente; 7-10 minutes. In a small bowl,
  combine miso paste and vinegar. Ladle in 2-3 Tbsp broth and stir until dissolved. Add back into pot, then take off
  heat.
\item
  Check seasoning and add salt or pepper as needed. Stir in parsley and serve.
\end{enumerate}

\hypertarget{notes-19}{%
\subsection{notes}\label{notes-19}}

Goes well with a crusty bread. Also stores well, but orzo will get plump and absorb
lots of the broth.

\hypertarget{turnip-red-lentil-stew}{%
\chapter{turnip red lentil stew}\label{turnip-red-lentil-stew}}

12.31.2017

turnip red lentil stew

ingredients \textbar{}
instructions \textbar{}
notes

\hypertarget{ingredients-20}{%
\subsection{ingredients}\label{ingredients-20}}

\begin{itemize}
\tightlist
\item
  1 sweet onion; diced
\item
  4 garlic cloves; minced
\item
  1 bunch of small turnips (I like hakurei); washed, trimmed, and quartered or 4 large turnips; peeled and 1/2" dice
\item
  1 can tomato puree
\item
  4 cups vegetable broth and extra vegetable base concentrate
\item
  1 bunch chard (could substitute lacinato kale); washed, thick stems removed, and thinly sliced
\item
  1 cup red lentils; rinsed just before cooking
\item
  2 tbsp olive or neutral oil
\item
  1/2 tsp thyme (fresh or dried)
\item
  1/4 tsp dried sage (optional)
\item
  pinch of red pepper flakes (optional)
\item
  2 shallots; cut in half lengthwise and thinly sliced (optional; see recipe)
\item
  1/4 cup neutral oil (optional; see recipe)
\item
  1 pinch salt (optional; see recipe)
\end{itemize}

\hypertarget{instructions-20}{%
\subsection{instructions}\label{instructions-20}}

\begin{enumerate}
\def\labelenumi{\arabic{enumi}.}
\tightlist
\item
  Heat pan on medium flame then add oil and onion. Cook until onion begins to soften then add garlic and cook until
  both are softened through - just sweating the veggies, no need to try to brown here. Add turnips and toss. Add tomato
  sauce and veggie broth and stir to combine. If using large turnips, cook for 15 minutes to get out any bitterness.\\
  Add lentils and stir again.
\item
  If making the optional crispy, caramelized shallots, add shallots and oil to a pan on medium low. After a few
  minutes the shallots should be sizzling as they release their moisture. Cook while stirring occasionally and continue
  with the stew steps below. About 20 minutes in the shallots should be turning golden. Remove them from the oil and
  spread out on a paper towel lined plate and sprinkle with salt.
\item
  Bring soup to a boil then reduce to a simmer and cook for about 20 minutes stirring occasionally. Around then the
  lentils will be a consistent color and texture though and will plump up to fill the pot all of the sudden. At this
  point stir in the herbs if you're using them and the greens and let cook for a couple minutes.
\item
  Taste the soup- if it lacks body, add vegetable broth base 1/2 tsp at a time, or maybe a pinch of salt if it just
  needs more seasoning.
\end{enumerate}

\hypertarget{notes-20}{%
\subsection{notes}\label{notes-20}}

Serve with a crack of pepper and/or a pinch of red pepper flakes.

\hypertarget{vegan-pumpkin-chocolate-cookies}{%
\chapter{vegan pumpkin chocolate cookies}\label{vegan-pumpkin-chocolate-cookies}}

12.22.2019

vegan pumpkin cholocate chip cookies

ingredients \textbar{}
instructions \textbar{}
notes

\hypertarget{ingredients-21}{%
\subsection{ingredients}\label{ingredients-21}}

\begin{itemize}
\tightlist
\item
  1/2 cup coconut oil; solid (be sure it's not melted)
\item
  3/4 cups dark brown sugar; packed
\item
  1/2 cup granulated sugar
\item
  2 tsp vanilla extract
\item
  2 Tbsp maple syrup
\item
  2 Tbsp unsweetened coconut milk (almond or soy milk will also work)
\item
  2/3 can pumpkin puree (not pumpkin pie mix)
\item
  2 1/4 cups all-purpose flour; unpacked
\item
  1 tsp baking soda
\item
  1/2 tsp salt
\item
  2 tsp pumpkin pie spice (optional)
\item
  1/2 - 1 cup chocolate chips
\item
  coarse salt
\end{itemize}

\hypertarget{instructions-21}{%
\subsection{instructions}\label{instructions-21}}

\begin{enumerate}
\def\labelenumi{\arabic{enumi}.}
\tightlist
\item
  Preheat oven to 375F. Line a large baking sheet with parchment paper; set aside.
\item
  In the bowl of a stand mixer fitted with the paddle attachment, or in a large bowl using a handheld electric mixer,
  beat the coconut oil, both sugars, and vanilla on medium-speed until smooth; about 2 minutes. Add in the molasses,
  coconut milk and pumpkin puree and beat on low speed until well combined. Turn mixer off.
\item
  In a separate bowl combine the flour, baking soda, salt, and pumpkin pie spice; whisk well to combine.
\item
  Add the dry ingredients into the wet mixture and, with the mixer on low speed, beat until ingredients are combined.
  The batter will be very thick! Fold in 1 and 1/4 cups of the chocolate chips.
\item
  Scoop three tablespoon sized mounds of dough onto the prepared cookie sheet, leaving a few inches between each
  cookie. Bake for 9 to 10 minutes, or until the edges are golden and the centers are soft but set. Press remaining
  chocolate chips on top of warm cookies, and sprinkle with sea salt, if using. Cool cookies on the baking sheet for 30
  minutes before transferring them to a cooling rack.
\end{enumerate}

\hypertarget{notes-21}{%
\subsection{notes}\label{notes-21}}

Jess' creation and favorite desert. These are somewhere between a cookie and a muffin top.

\hypertarget{coconut-baked-oatmeal}{%
\chapter{coconut baked oatmeal}\label{coconut-baked-oatmeal}}

9.18.2018

coconut baked oatmeal

ingredients \textbar{}
instructions \textbar{}
notes

\hypertarget{ingredients-22}{%
\subsection{ingredients}\label{ingredients-22}}

\begin{itemize}
\tightlist
\item
  1 Tbsp coconut oil (or other neutral oil to grease baking pan)
\item
  3 ripe bananas
\item
  2 eggs
\item
  1/3 cup shredded, unsweetened coconut
\item
  1/4 cup brown sugar
\item
  1 tsp vanilla extract
\item
  1 tsp baking powder
\item
  1/2 tsp kosher salt
\item
  1 can coconut milk
\item
  2 1/4 cup rolled oats (not quick cooking)
\item
  bluberries, strawberries, raspberries, blackberries; washed (optional)
\item
  coconut whipped cream (optional)
\end{itemize}

\hypertarget{instructions-22}{%
\subsection{instructions}\label{instructions-22}}

\begin{enumerate}
\def\labelenumi{\arabic{enumi}.}
\tightlist
\item
  Preheat the oven to 375F (without convection) and melt coconut oil (or replacement) to grease baking dish.
\item
  Smash bananas in large mixing bowl. Combine all ingredients up to rolled oats and mix well. Spoon in oats and
  stir to combine.
\item
  Bake for 45 minutes. Remove and let cool for a couple minutes before digging in. Optionally, top with fruit and
  coconut whipped cream.
\end{enumerate}

\hypertarget{notes-22}{%
\subsection{notes}\label{notes-22}}

Can be served warm or cold. Keeps for a few days in the fridge.

\hypertarget{coconut-rice}{%
\chapter{coconut rice}\label{coconut-rice}}

6.21.2019

coconut rice

ingredients \textbar{}
instructions \textbar{}
notes

\hypertarget{ingredients-23}{%
\subsection{ingredients}\label{ingredients-23}}

\begin{itemize}
\tightlist
\item
  2 cup jasmine rice; thoroughly rinsed
\item
  1 14 oz can of coconut milk
\item
  1.5 cup water
\item
  1 tsp salt
\item
  3 Tbsp shredded coconut; plus more, toasted for a garnish if desired
\end{itemize}

\hypertarget{instructions-23}{%
\subsection{instructions}\label{instructions-23}}

\begin{enumerate}
\def\labelenumi{\arabic{enumi}.}
\tightlist
\item
  Combine in a rice cooker and stir.
\item
  Cook on sweet setting, or just the white setting if sweet is not an option.
\item
  Fluff and stir to combine the shredded coconut which will have floated to the top.
\end{enumerate}

\hypertarget{notes-23}{%
\subsection{notes}\label{notes-23}}

High quality coconut milk makes a massive difference. I really like Thai Kitchen brand, and dislike Trader Joes.

\hypertarget{daily-daal}{%
\chapter{daily daal}\label{daily-daal}}

2.9.2018

daily daal

ingredients \textbar{}
instructions \textbar{}
notes

\hypertarget{ingredients-24}{%
\subsection{ingredients}\label{ingredients-24}}

\begin{itemize}
\tightlist
\item
  8 oz red lentils (about 1 heaping cup)
\item
  2 Tbsp neutral cooking oil (e.g., avocado)
\item
  10 black peppercorns
\item
  5 cloves
\item
  1 onion; small dice
\item
  2.5" ginger; finely mined or grated
\item
  5 garlic cloves; peeled
\item
  1 tsp chili powder
\item
  1 scant tsp coriander
\item
  1/2 tsp turmeric
\item
  1 tsp salt
\item
  1 14oz can plum tomatoes
\item
  cayenne pepper (optional)
\item
  handful of cilantro; rinsed and roughly chopped (optional)
\item
  one recipe basmati rice
\item
  lemon (optional)
\end{itemize}

\hypertarget{instructions-24}{%
\subsection{instructions}\label{instructions-24}}

\begin{enumerate}
\def\labelenumi{\arabic{enumi}.}
\tightlist
\item
  Rinse red lentils until water runs clear. Put in tall pot and cover with 2.5 cups cold water.\\
  Bring to a boil then reduce heat, cover, and simmer for 10-15 minutes.
\item
  While lentils are cooking, heat the oil in a new pan over medium-low heat. When hot, add peppercorns and
  cloves and cook until fragrant - about 30 seconds. Add onions and cook while stirring often for 8-10 minutes
  or golden. Add the ginger and crush in the garlic (I use a press) and cook for 2-3 minutes until fragrant but
  not burnt. Tip in spices and salt and stir to coat.
\item
  Add can of tomatoes, crushing with your hands as they go in (granny style) or crush with wooden spoon in pan.\\
  Bring to a boil then simmer for 5 minutes. Cover and simmer 5 more.
\item
  Once the sauce has darkened and thickened a little bit start adding in the lentils with a slotted spoon.\\
  If it's too thick, thin with some of the lentil cooking liquid. I found that mine typicaly doesn't need any
  additional liquid.
\item
  Simmer on low heat for 10 more minutes, uncovered if looking to thicken the sauce. Taste and adjust seasoning.\\
  Add cayenne if it could use some spice. Add lemon juice for acid if needed.
\item
  Serve with basmati rice or Jeff's naan and a sprinkle of chopped cilantro.
\end{enumerate}

\hypertarget{notes-24}{%
\subsection{Notes}\label{notes-24}}

Might want to remove cloves and peppercorns before serving if they're too powerful.

\hypertarget{food-processor-pasta}{%
\chapter{food processor pasta}\label{food-processor-pasta}}

7.21.2017

food processor pasta dough

ingredients \textbar{}
instructions \textbar{}
notes

\hypertarget{ingredients-25}{%
\subsection{ingredients}\label{ingredients-25}}

\begin{itemize}
\tightlist
\item
  3 eggs
\item
  1/2 tsp salt
\item
  2 cups all purpose flour
\item
  1 Tbsp olive oil
\end{itemize}

\hypertarget{instructions-25}{%
\subsection{instructions}\label{instructions-25}}

\begin{enumerate}
\def\labelenumi{\arabic{enumi}.}
\tightlist
\item
  Pulse salt and flour in a food processor.
\item
  Add eggs and olive oil to bowl and process for 30-60 seconds. The contents should come together into a large ball.\\
  If the dough isn't coming coming together add water 1 tbsp at a time and process again for 15 seconds.
\item
  Once you have a ball, plop the dough onto a cutting board and knead until the ball is relatively smooth and soft.\\
  This is typically a minute or two.
\item
  Cover the dough with plastic wrap and let rest for 30 minutes. Once rested it is ready to be rolled into whatever style pasta you'd like
\end{enumerate}

\hypertarget{notes-25}{%
\subsection{notes}\label{notes-25}}

For a more decadent pasta, replace one whole egg with two yolks.

\hypertarget{hummus-caesar-dressing}{%
\chapter{hummus caesar dressing}\label{hummus-caesar-dressing}}

6.12.2019

hummus caesar dressing

ingredients \textbar{}
instructions \textbar{}
notes

\hypertarget{ingredients-26}{%
\subsection{ingredients}\label{ingredients-26}}

\begin{itemize}
\tightlist
\item
  1/2 cup hummus (roasted garlic preferred)
\item
  1 tsp spicy whole grain mustard
\item
  1/2 tsp lemon zest (1 lemon's worth)
\item
  1 Tbsp capers; finely minced
\item
  1 heaping Tbsp caper brine
\item
  3 Tbsp garlic; minced (4-5 small cloves cloves)
\item
  1 pinch salt
\item
  couple cracks of ground pepper
\item
  1-2 Tbsp olive oil
\item
  1 tsp maple syrup
\item
  water (optional); to thin sauce
\item
  lemon juice (optional); to taste as needed
\end{itemize}

\hypertarget{instructions-26}{%
\subsection{instructions}\label{instructions-26}}

\begin{enumerate}
\def\labelenumi{\arabic{enumi}.}
\tightlist
\item
  Stir everything together.
\item
  Thin slightly with water to get desired consistency.
\item
  Add lemon juice as needed. I find this often isn't necessary with good caper brine.
\item
  Can be served right away but does well with a sit for the flavors to mingle -- maybe 30 minutes.
\end{enumerate}

\hypertarget{notes-26}{%
\subsection{notes}\label{notes-26}}

Really good massaged into crunchy greens; e.g., romaine hearts, lacinato kale.

\hypertarget{jeff-naan}{%
\chapter{jeff naan}\label{jeff-naan}}

2.2.2016

jeff's naan

ingredients \textbar{}
instructions \textbar{}
notes

\hypertarget{ingredients-27}{%
\subsection{ingredients}\label{ingredients-27}}

\begin{itemize}
\tightlist
\item
  2 tsp dry active yeast
\item
  1 tsp sugar
\item
  1/2 cup water
\item
  2 1/2 - 3 cups all-purpose flour, divided
\item
  1/2 tsp salt
\item
  1/4 cup olive oil
\item
  1/3 cup plain yogurt (greek works as well)
\item
  1 large egg
\end{itemize}

\hypertarget{instructions-27}{%
\subsection{instructions}\label{instructions-27}}

\begin{enumerate}
\def\labelenumi{\arabic{enumi}.}
\tightlist
\item
  In a small bowl, combine the yeast, sugar and water. Stir to dissolve then let sit for a few minutes or until it is
  frothy on top. Once frothy, whisk in the oil, yogurt, and egg until evenly combined.
\item
  In a separate medium bowl, combine 1 cup of the flour with the salt. Next, pour the bowl of wet ingredients
  to the flour/salt mixture and stir until well combined. Continue adding flour, a half cup at a time, until you can
  no longer stir it with a spoon (about 1 to 1.5 cups later). I find this is almost always 2.5 cups of flour total
  when using king arthur all-purpose flour.
\item
  At that point, turn the ball of dough out onto a lightly floured surface and knead the ball of dough for about
  3 minutes, adding small amounts of flour as necessary to keep the dough from sticking. You'll end up using between
  2.5 to 3 cups flour total. The dough should be smooth and very soft but not sticky. Avoid adding excessive amounts
  of flour as you knead, as this can make the dough too dry and stiff.
\item
  Loosely cover the dough and let it rise until double in size (about 1 hour). After it rises, gently flatten the
  dough into a disc and cut it into 8 equal pieces. Shape each piece into a small ball.
\item
  Heat a large, heavy bottomed skillet over medium heat (I prefer cast iron for this). Working with one ball at a time,
  roll it out until it is about 1/4 inch thick or approximately 6 inches in diameter. Place the rolled out dough onto
  the hot skillet and cook until the bottom is golden brown and large bubbles have formed on the surface. Flip the dough
  and cook the other side until golden brown as well. Stack the cooked flat bread on a plate and cover with a towel to
  keep warm as you cook the remaining pieces. Serve plain or brushed with melted butter and sprinkled with
\end{enumerate}

\hypertarget{notes-27}{%
\subsection{notes}\label{notes-27}}

This is pretty much just the budget bytes homemade naan recipe. These keep well stored in mason jars then reheated on
a hot pan like they were cooked.

\hypertarget{jess-bowl}{%
\chapter{jess bowl}\label{jess-bowl}}

4.11.2018

jess bowl

ingredients \textbar{}
instructions \textbar{}
notes

\hypertarget{ingredients-28}{%
\subsection{ingredients}\label{ingredients-28}}

\begin{itemize}
\tightlist
\item
  neutral oil
\item
  1 3/4 cup brown rice
\item
  1/4 cup wild rice; cooked
\item
  2 large sweet potatoes; 3/4" dice
\item
  2 delicata squash; cut into half moons
\item
  1 recipe miso braised greens
\item
  8 oz crimini mushrooms; stems removed and washed, cut in half
\item
  2 tbsp neutral oil
\item
  kosher salt
\item
  1 avocado; sliced
\item
  sriracha to taste
\item
  toasted sesame oil to taste
\item
  soy sauce or tamari to taste
\item
  1 tsp white sesame seeds
\item
  1 tsp black sesame seeds
\item
  fried egg (optional)
\end{itemize}

\hypertarget{instructions-28}{%
\subsection{instructions}\label{instructions-28}}

\begin{enumerate}
\def\labelenumi{\arabic{enumi}.}
\tightlist
\item
  Cook rice in a rice cooker on brown setting or on the stove.
\item
  Heat oven to 400F. Peel sweet potatoes and cut into 3/4" cubes. Toss in a 1-2 Tbsp neutral oil with a
  sprinkle of salt. Cut off top of delicata squash, slice length-wise, scoop out inards, and slice into 1/2" wide
  half moons. Again, toss in a neutral oil with a sprinkle of salt. Using one large or two smaller trays, spread
  out sweet potatoes and delicata squash. Roast for 20 minutes, until browning underneath. Flip and toss. Roast for
  another 10-20 minutes. Sweet potatoes should be bronzed and tender and squash should be browning at edges. Toss
  chunks around one more time if it looks like they are cooking unevenly.
\item
  While potatoes and squash are cooking, heat a pan to medium-low heat. Add 2 Tbsp neutral oil and crimini mushrooms
  to the pan, cooking until tender. While mushrooms are cooking, prepare japanese braised greens. Jess prefers using
  lacinato kale for this specific recipe, as opposed to chard. Throw cooked mushrooms into the pan of greens, stir, and
  cook together on low for another minute. Remove from heat and set aside.
\item
  Prepare bowls. Before adding vegetables, mix a dash of toasted sesame oil into the cooked rice. Pile on vegetables
  and top with a fried egg (optional). Add tamari/soy sauce and siracha to taste.
\end{enumerate}

\hypertarget{notes-28}{%
\subsection{notes}\label{notes-28}}

Best rice bowl ever. Rice can be cooked on a timer so that it can soak for longer, up to 12 hours.

\bibliography{book.bib,packages.bib}

\end{document}
